\documentclass[12pt]{report}

\usepackage[pdftex]{graphicx}
\usepackage[normalem]{ulem}
\usepackage[utf8]{inputenc}
\usepackage[english,frenchb,francais]{babel}
\usepackage[T1]{fontenc}
\usepackage{hyperref}

\begin{document}

  \begin{titlepage}
    \centering
        \vfill
        {\rule{\linewidth}{.5pt}
        \huge
            Optimisation par algorithme génétique\\
          \large
            Influence du paramétrage\\
          \rule{\linewidth}{.5pt}
            \vskip2cm
            Swan Launay - Gabriel Vaubaillon\\
        }
        \vfill
        \includegraphics[height=1.7cm]{logo/polytech.jpg}
        \hfill
        \includegraphics[height=1.7cm]{logo/usmb.png}
  \end{titlepage}

  \chapter{Préambule}
    \section{Remerciements}
      Ce rapport représente l'aboutissement du projet de 40 heures qui s'est déroulé entre octobre 2018 et avril 2019. Nous tenons tout d'abord à remercier Gilles Fraisse pour son soutien, sa pédagogie, mais aussi pour le temps qu'il a passé à nous accompagner.
      
      Nous tenons aussi à remercier l'Université Savoie Mont Blanc (USMB) pour la mise à disposition des documents nécessaires à la réalisation de ce projet par le biais de la bibliothèque universitaire.

      Nous remercions enfin Polytech Annecy - Chambéry pour nous avoir permis d'effectuer ce projet sur notre temps de travail universitaire et plus globalement pour nous avoir proposé un travail de ce type.

    \section{Résumé}
      L'optimisation par algorithme génétique permet d'obtenir de bonnes approximations de résolutions pour différents problèmes (avec un ou plusieurs objectifs). De façon générale, les algorithmes génétiques sont construits de telle façon à ce que l'on puisse faire varier certains paramètres. C'est ces mêmes paramètres qui vont déterminer la qualité et la fiabilité du résultat, mais aussi qui vont faire varier de façon plus ou moins significative le temps de résolution. Il s'agit alors de trouver un juste milieu entre le temps résolution et la fiabilité du résultat.

  \chapter{Introduction}
      Parler de l'inteet de l'opti dans le monde de l'ingienerie
  \tableofcontents

  \chapter{Optimisation à objectif unique}



  \chapter{Optimisation à objectifs multiples}

  \chapter{Conclusion}

  \chapter{Bibliographie}


\end{document}
